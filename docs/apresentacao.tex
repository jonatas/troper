\documentclass{beamer} 
\usepackage[brazil]{babel}
\usepackage[T1]{fontenc}
\usepackage[utf-8]{inputenc}
\usepackage{listings}
\lstset{language=ruby,caption=Exemplo,label=Ruby, numbers=left, frame=single} 
%\usecolortheme{seahorse}

\begin{document} 
\begin{frame} 
\frametitle{Troper} 
\begin{itemize} 
\item O sistema
\item O plugin
\item As ferramentas 
\item As métodologias
\item Considerações Finais
\end{itemize} 
\end{frame} 

\begin{frame} 
\frametitle{Troper - O Sistema Hospedeiro} 
\begin{itemize} 
  \item Baseado nos modelos já existentes no sistema
  \item Usa as regras do negócio já existentes
\end{itemize} 
\end{frame} 

\begin{frame}[fragile]
\frametitle{Troper - O Sistema Hospedeiro} 
\begin{lstlisting}[caption=Modelo ActiveRecord]
# app/models/person.rb
class Person < ActiveRecord::Base
  has_many :addresses
end
# app/models/address.rb
class Address < ActiveRecord::Base
  belongs_to :person 
end
\end{lstlisting}
\end{frame}

\begin{frame} 
\frametitle{Troper - A template Liquid padrão } 
\begin{itemize} 
  \item Usa as regras de negócio dos modelos
  \item O plugin entende estas regras através das \textit{reflections} do modelo
  \item Extensibilidade das tags para liquid
  \item Criação de formatadores e outros de forma simples 
\end{itemize} 
\end{frame} 

\begin{frame}[fragile]
\frametitle{Troper - Um exemplo Liquid} 
\begin{lstlisting}[language=html, caption=Template do Liquid]
<table>
  
    <tr>
      <td>{{ person.name }}</td>
      <td>{{ person.email | link_to_mail }}</td>
      <td>{{ person.phone | format_phone }}</td>
     </tr>
         
</table>
\end{lstlisting}
\end{frame}

\begin{frame}[fragile]
\frametitle{Troper - Um exemplo do Troper} 
\begin{lstlisting}[language=ruby, caption=Usando o Troper]
report = Troper::Report.new("people")
report.columns.add "addresses"
puts report.template_to_resource
\end{lstlisting}
\end{frame}

\begin{frame}[fragile]
\frametitle{Troper - Consultando a template} 
\begin{lstlisting}[language=html, caption=Resultado do puts anterior]
<table id="people">
  <tr>
     <td>Name</td><td>Email</td><td>Phone</td><td>Addresses</td>
  </tr>
  
  <tr><td>{{ person.name }}</td>
      <td>{{ person.email }}</td>
      <td>{{ person.phone }}</td>
      <td>
      <table id="addresses">
      <tr><td>Description</td><td>City</td><td>State</td></tr>
      
        <tr><td>{{ address.description }}</td>
            <td>{{ address.city }}</td>
            <td>{{ address.state }}</td></tr>
      
      </table></td>
  </tr>
  
</table>
\end{lstlisting}
\end{frame}

\begin{frame}[fragile]
\frametitle{Troper - Consultando a saída do relatório} 
\begin{lstlisting}[language=html, caption=report.output]
>> puts report.output
<table id="people">
  <tr><td>Name</td><td>Email</td><td>Phone</td><td>Addresses</td></tr>
  <tr><td>Jonatas Davi Paganini</td>
  <td>jonatasdp@gmail.com</td>
  <td>46 9911 7879</td><td>
  <table id="addresses">
  <tr><td>Description</td>
      <td>City</td><td>State</td></tr>
  <tr><td>rua pernambuco, 123, fco beltrao - pr</td><td></td><td></td></tr>
  <tr><td>avenida julio assis, fco beltrao - pr</td><td></td><td></td></tr>
</table>...
\end{lstlisting}
\end{frame}

\begin{frame}[fragile]
\frametitle{Troper - Cadê os filtros?} 
\begin{lstlisting}[language=ruby, caption=Adicionando filtros]
report.filters << "person.addresses != empty"
\end{lstlisting}

Na template será adicionada um \textbf{if}:

\begin{lstlisting}[language=html, caption=Exibindo na template]
  
    
       <tr><td>{{ person.name }}</td> #...
\end{lstlisting}
\end{frame}

\begin{frame}[fragile]
\begin{lstlisting}[language=ruby, caption=Adicionando formatadores]
col = report.columns[:description]
col.formats << "truncate: 50"
\end{lstlisting}

Adicionando um \textbf{método} na template:

\begin{lstlisting}[language=html, caption=Exibindo na template]
# ...
  <td>{{ record.description | truncate: 50 }}</td>
#...
\end{lstlisting}

\end{frame}

\begin{frame} 
\frametitle{Troper - Ferramentas}
  \begin{itemize} 
    \item Rails
    \item ExtJS 
    \item Liquid Templates 
    \item Terminal Unix 
    \item GIT 
    \item Vim 
    \item Cucumber
    \item Rake 
    \item GitHub 
    \item LaTeX
  \end{itemize} 
\end{frame}

\begin{frame}
\frametitle{Troper - As métodologias}
  \begin{itemize} 
    \item Behaviour Development Driven - Cucumber
    \item DRY - Don't repeat yourself 
    \item Convention Over Configuration
    \item Prototipação 
  \end{itemize} 
\end{frame}
\begin{frame}[fragile]
\frametitle{Troper - Estórias do usuário}
\begin{verbatim}
Funcionalidade: Gerar relatorio a partir de uma tabela.
  Cenario: Criar um relatorio usando uma tabela como fonte de dados
     Dado que posso ver a tabela pessoas como relatorio
     Quando olhar para as colunas desse relatorio
     Então deve ser igual: nome, email, telefone

     Quando eu olhar para a coluna 'nome'
       Então o titulo deve ser igual a 'Nome'
       E o recurso deve ser igual a 'pessoa.nome'
       E a template para o recurso deve ser \
           igual a '{{ pessoa.nome }}'
\end{verbatim}
\end{frame}

\begin{frame}
\frametitle{Troper - Padrões de Projeto}
  \begin{itemize} 
    \item Forwardable
    \item Command Pattern 
  \end{itemize} 
\end{frame}

\begin{frame} 
\frametitle{Troper - Não fez uso das tecnologias }
  \begin{itemize} 
    \item Windows 
    \item IDE
    \item Banco de dados para armazenamento de objetos
  \end{itemize} 
\end{frame} 

\begin{frame} 
\frametitle{Troper - Apoia!}
  \begin{itemize} 
    \item Ruby 
    \item Open Source 
    \item Linux 
    \item GIT 
  \end{itemize} 
\end{frame} 
\end{document} 
