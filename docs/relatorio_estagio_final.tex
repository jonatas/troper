%\documentclass[12pt]{article}
\documentclass[espaco=simples,appendix=Name]{abnt}
\usepackage{abntex}
\usepackage[brazil]{babel}
\usepackage[T1]{fontenc}
\usepackage[utf-8]{inputenc}
\usepackage{hyperref}
\usepackage{times}
\usepackage{listings}
\usepackage[dvips]{graphicx}
\usepackage[num]{abntcite}      % citacoes do abntex
\usepackage{tabela-simbolos}    % tabelas de simbolos do abntex
\usepackage{dsfont}             % fonte
\usepackage{fancyvrb}

\citeoption{abnt-full-initials=yes}
\lstset{language=ruby,caption=Exemplo,label=Ruby, numbers=left, frame=single} 

\title{Troper: relatório final de estágio}

\author{Jônatas Davi Paganini}

\date{novembro de 2009}

\begin{document}
\maketitle

 Um ano se passou e aqui está o sistema desenvolvido. Após muitos meses de estudo, análise de outros frameworks, decisões de design e arquitetura, está pronto o primeiro \textit{release} do sistema. As lições aprendidas que pude tirar foram em vários aspectos:

\textbf{Simplicidade} é um dos princípios do Extremme Programming. Através da expressividade da linguagem Ruby e das ferramentas já existentes, é possível se conseguir um ótimo resultado com poucas linhas de código.

\textbf{Estudo da melhor forma de executar} tem sido um dos desafios neste projeto. Cada detalhe dos frameworks utilizados, consegue trazer um recurso exato de forma simples e direta. Por exemplo, em uma necessidade de automaticamente reconhecer os recursos de banco de dados do sistema, é preciso reconhecer os padrões e implementar o método to\_liquid para cada objeto que deseja ter acesso na visão. Esta é uma forma interessante de protejer o banco de dados e qualquer tipo de \textit{sql injection}. Para isso, se deseja ter acesso a um atributo, este atributo deve responder pelo método to\_liquid. Por exemplo, em uma classe cliente, com um atributo nome, este atributo deve responder pelo método to\_liquid. No próprio framework liquid, existe um método auxiliar que gera os métodos to\_liquid via meta-programação, precisando apenas usar a seguinte sintaxe: 

\begin{lstlisting}[caption=Implementação do método to\_liquid através de um método helper]
class Pessoa < ActiveRecord::Base
  liquid_methods :nome, :telefone, :enderecos
end
\end{lstlisting}

Desta forma, se este código for usado em uma classe que contenha estes atributos ou métodos, ele irá reconhecer a fonte de dados e os métodos acessíveis. Com esta forma simples de declaração, agora apenas é necessário descobrir seguramente quais atributos podem ir para a visão, ou seja, podem ser vistos na template do liquid. 

O uso de testes automatizados ajudou muito no período em que estava sendo desenvolvido o núcleo do \textit{plugin}. Inspirado nos \textit{frameworks} já existentes, nos princípios do \textit{Rails}: \textit{convention over configuration}(convenção ao invés de configuração) e \textit{DRY: Don't repeat yourself}(não se repita) foi possível testar o comportamento da ferramenta perante um sistema de exemplo, foi possível documentar e escrever o \textit{plugin} ao mesmo tempo. Através de técnicas de \textit{Behaviour Driven Development}(Desenvolvimento Orientado a Comportamento) e \textit{frameworks} como o \textit{Cucumber}, foi possível documentar e testar a funcionalidade que o \textit{plugin} oferecia. 

Com poucas linhas de código, os testes desenvolvidos nas estórias trouxeram um exemplo de como o sistema funciona. Cada linha da estória sugere uma situação específica de uso do \textit{plugin}. Através de exemplos, é possível qualquer desenvolvedor que use o \textit{plugin} consiga também trazer novas funcionalidades e contribuir com o código livre.

Durante o desenvolvimento do estágio, houve a necessidade de automatização dos processos de relatórios de estágio e relatório mensal da empresa. Neste momento, houve o primeiro desafio do \textit{plugin} de usar as templates do framework liquid. Através de apenas um arquivo de texto plano e algumas convenções, o arquivo acompanhamento.textile divide as tarefas e históricos para apresentação do relatório mensal da empresa e o relatório de estágio.

Por padrão, o sistema Troper adotou a visão que todos atributos de um objeto e seus relacionamentos devem ser fornecidos para a template liquid. Desta forma foi preciso apenas navegar pelos modelos fornecidos pelo sistema base e apartir de cada modelo foi invocado o método \textbf{liquid\_methods} com todos os atributos e relacionamentos do banco de dados.

\begin{lstlisting}[caption=Implementação do método to\_liquid através da meta-programação]

for model in self.models
  if not model.instance_method_already_implemented? "to_liquid"
    attrs = model.columns.collect(&:name) +
            model.reflections.keys
    model.class_eval { liquid_methods *attrs } 
  end
end
\end{lstlisting}


O \textit{plugin} também tem \textbf{segurança}. No código acima, o método que "adivinha" os atributos, está limitado a sobrescrição do próprio sistema(linha 2). Ou seja, a implementação só será feita, se o desenvolvedor não o fizer.

Enquanto todos os preparativos de background estavam sendo programados, o conhecimento sobre a interface estava sendo adquirido. Através de livros, fóruns e exemplos, foi possível criar uma interface muito semelhante a proposta em rascunhos. A ferramenta usada para construir a interface, permitiu que fosse totalmente codificado em javascript e html.

A interface do sistema apenas faz referência a um arquivo html. O controlador responsável por trocar informações sobre os modelos é injetado no sistema base apenas para responder sobre quais elementos pode exibir ou não.

A linguagem de programação utilizada neste projeto, teve sim, um único aspecto negativo: a falta de conhecimento dos professores sobre a linguagem e o contexto, tornando um pouco abstratas as tarefas e explicações sobre o que estava sendo realizado.

A gestão do projeto trouxe mais desenvolvimento de software para o projeto. Na tentativa de automatizar muitos processos burocráticos para documentação, o tempo do desenvolvimento foi compartilhado com as tarefas da gerência do projeto. A dificuldade de trabalhar com documentos formais trouxe ainda o estudo da ferramenta LaTeX e outros \textit{frameworks} como \textit{RedCloth, Textile, Rake, Rak} e ferramentas como \textit{Git, Shell Script} conseguiram integrar o processo de desenvolvimento e acompanhamento do orientador. Com todas essas ferramentas, a integração do sistema tornou-se uma mão cheia de pequenos \textit{frameworks} com funcionalidades bem definidas.

O processo de integração contínua no ambiente de desenvolvimento, é um fato esquecido, ou até mesmo desconhecido pelos professores. Mesmo não fazendo parte da ementa, este problema complica a gestão do projeto. A atitude tomada no projeto atual, trouxe atrasos ao projeto e também foi além do escopo acordado inicialmente. Em meio a esta dualidade, esta foi uma boa experiência e as duas partes têm suas compensações positivas. 

\end{document}
